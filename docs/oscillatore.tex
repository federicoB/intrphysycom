\documentclass[11pt,a4paper,twoside]{book}
\usepackage[italian]{babel}
\usepackage[left=2.5cm,top=3.5cm,bottom=3cm,right=2.5cm,nohead,nofoot]{geometry}
\begin{document}
\pagestyle{empty}
\parindent 0pt

{\bf  \begin{center}
  {\large \bf Esempio 2: equazione del moto armonico $ \ddot{x}+x=0 $  }        \end{center} }

\vskip 0.5 truecm \textbf{Metodo 1: Schema centrato del secondo
ordine.}\vskip 0.1 truecm L'equazione differenziale \'e del secondo
ordine e pu\'o essere risolta approssimandola con il metodo
centrato, come segue:

$$ x_{i+1}-2x_i+x_{i-1} +x_i\Delta^2=0 $$

Per far partire questo schema \'e necessario conoscere le due
condizioni iniziali $x_0$ ed $x_1$. La prima rappresenta il valore
della variabile x al tempo $t=0$, mentre la seconda pu\'o essere
cos\'i calcolata:

$$ x_1=x_0+ \dot{x}_0 \Delta + \ddot{x}_0 \Delta^2/2 $$
$$ x_1= x(0)(1-\Delta^2/2)+v(0)\Delta $$

L'equazione differenziale di partenza equivale ora a risolvere il
seguente sistema:
\begin{displaymath}
\left\{
  \begin{array}{ll}
    x_0=x(0), & \hbox{} \\
    x_1= x(0)(1-\Delta^2/2)+v(0)\Delta, & \hbox{} \\
    x_{i+2}= 2(1-\Delta^2/2)x_{i+1}-x_i, & \hbox{}
  \end{array}
\right. \end{displaymath}

Per risolvere l'equazione alle differenze, si assume $ v_i=p^i $

$$ p^2-2(1-\Delta^2/2)p+1=0 $$

$$ p= (1-\Delta^2/2)\pm i\Delta (1-\Delta^2/4)^{1/2} $$

$$ |p|^2 = (1-\Delta^2/2)^2 + \Delta^2-\Delta^4/4 = 1 $$

 questo assicura che la soluzione non cresce ne decade.


\vskip 1.0 truecm \textbf{Metodo 2: Schema Runge Kutta  del secondo
ordine. } \vskip 0.1 truecm In questo caso, l'equazione
differenziale di partenza (che era del secondo ordine) viene
riscritta come un sistema di due equazioni differenziali del primo
ordine:
\begin{displaymath}
\left\{
  \begin{array}{ll}
    \dot{x}=+v, & \hbox{} \\
    \dot{v}=-x, & \hbox{}
  \end{array}
\right. \end{displaymath}



Un passo temporale $\Delta$ \'e composto di due sotto passi:
$$ \hat{x}=x+v\Delta/2 $$
$$ \hat{v}=v-x\Delta/2 $$

$$ x=x+\hat{v}\Delta $$
$$ v=v-\hat{x}\Delta $$

eliminando le variabili intermedie si ottiene:

$$ x_{i+1}=x_i(1-\Delta^2/2)+v_i\Delta $$
$$ v_{i+1}=v_i(1-\Delta^2/2)-x_i\Delta $$

$$ x_{i+2}-2(1-\Delta^2/2)x_{i+1} +(1+\Delta^4/4)x_i=0 $$

Per risolvere l'equazione alle differenze, si assume $ v_i=p^i $:

$$ p^2-2(1-\Delta^2/2)p+(1+\Delta^4/4)=0 $$

$$ p= (1-\Delta^2/2)\pm i\Delta $$
$$ |p|^2 = (1+\Delta^4/4) > 1 $$

questo dimostra che la nostra soluzione lentamente cresce.

\vskip 1.0 truecm

\textbf{Metodo 3: Schema simplettico del secondo ordine. } \vskip
0.1 truecm Anche in questo caso la funzione di partenza equivale a:
\begin{displaymath} \left\{
  \begin{array}{ll}
    \dot{x}=+v, & \hbox{} \\
    \dot{v}=-x, & \hbox{}
  \end{array}
\right. \end{displaymath}



Un passo temporale $\Delta$ , per\'o, questa volta \'e composto di
tre sotto passi:
$$ x=x+\dot{x}\Delta/2=x+v\Delta/2 $$
$$ v=v+\dot{v}\Delta=v-x\Delta $$
$$ x=x+\dot{x}\Delta/2=x+v\Delta/2 $$


% $$ \hat{x}=x+v\Delta/2 $$
% $$ v=v-\hat{x}\Delta $$
% $$ x=x+v\Delta/2 $$

$$ x_{i+0.5}=x_i+v_i\Delta/2 $$
$$ v_{i+1}=v_i-x_{i+0.5}\Delta $$
$$ x_{i+1}=x_{i+0.5}+v_{i+1}\Delta/2 $$

eliminiamo $ x_{i+0.5} $

$$ v_{i+1}= (1-\Delta^2/2)v_i-x_i\Delta $$
$$ x_{i+1}= x_i+(v_i+v_{i+1})\Delta/2 $$

eliminiamo x

$$ v_{i+2}-v_{i+1}= (1-\Delta^2/2)(v_{i+1}-v_i)-(x_{i+1}-x_i)\Delta $$
$$ x_{i+1}-x_i=(v_i+v_{i+1})\Delta/2 $$

$$ v_{i+2}-v_{i+1}= (1-\Delta^2/2)(v_{i+1}-v_i)-(v_i+v_{i+1})\Delta^2/2 $$

$$ v_{i+2}-2(1-\Delta^2/2)v_{i+1}+v_i=0 $$

Quello che alla fine si ottiene \'e la stessa equazione del primo
caso.


\end{document}
