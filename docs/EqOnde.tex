\documentclass[11pt]{report}
\usepackage[italian]{babel}
\begin{document}
\pagestyle{empty}
\parindent 0pt
\hsize = 14.0truecm

\newcommand{\dx}{{\partial \over \partial x}}
\newcommand{\dxx}{{\partial^2 \over \partial x^2}}
\newcommand{\dt}{{\partial \over \partial t}}


{\bf  \begin{center} {\large \bf Eq. Navier Stokes 1D }  \end{center} }



\begin{eqnarray*}
   && \dt \rho        = -\dx (\rho u)                              \\
   && \dt u  +u \dx u = -{ 1 \over \rho} \dx p +\nu \dxx u         \\
   && \dx p = c^2 \dx \rho
\end{eqnarray*}


\begin{eqnarray}
   && \dt \rho        = -\dx (\rho u)                               \label{1} \\
   && \dt u  +u \dx u = -{ c^2 \over \rho} \dx \rho +\nu \dxx u     \label{2}
\end{eqnarray}



Consideriamo la dissipazione trascurabile ($\nu=0$) e linearizziamo intorno ad uno stato di quiete ($U_0=0$).

\begin{eqnarray*}
   u &=& U_0 + \grave{u}                              \\
   \rho &=& \rho_0 + \grave{\rho}
\end{eqnarray*}


\begin{eqnarray*}
   \dt \grave{\rho}        &=& -\rho_0\dx \grave{u}                               \\
   \dt \grave{u} &=& -{ c^2 \over \rho_0} \dx \grave{\rho}
\end{eqnarray*}

Riscalando tutte le variabili, dipendenti ed indipendenti possiamo ottenere:

\begin{eqnarray*}
   \dt \rho        &=& -\dx u                               \\
   \dt u &=& - \dx \rho
\end{eqnarray*}


Questo sistema si presta ad essere discretizzato su una griglia "staggered"
e risulta semplice se assumiamo condizioni al contorno periodiche.


\end{document}
