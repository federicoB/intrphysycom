\documentclass[11pt,a4paper,twoside]{book}
\usepackage[italian]{babel}
\usepackage[left=2.5cm,top=3.5cm,bottom=3cm,right=2.5cm,nohead,nofoot]{geometry}
\begin{document}
\pagestyle{empty}
\parindent 0pt
%\hsize = 15.0 truecm
%\large
{\bf  \begin{center}
  {\large \bf Esempio 1: esponenziale $ \dot{x}=x $  }        \end{center} }
%\input colori


Si considera una funzione nota $ \dot{x}=x $,con $x_0$ condizione
iniziale al tempo $t = 0$:

\begin{displaymath}
\left\{  \begin{array}{ll}
    \dot{x}=x & \hbox{equazione differenzioale} \\
     x(0)=x_0 & \hbox{condizione iniziale}
  \end{array}
\right.
\end{displaymath}



Sappiamo che la soluzione generale di questa equazione differenziale
\'e $ x(t)= x_0 e^t $, ma si vuole ora ricavarla utilizzando due
metodi matematici differenti:

\vskip 1.5 truecm

\textbf{Metodo di Eulero - schema del primo ordine.}

Si approssima la derivata della funzione con il rapporto
incrementale $\dot{f}=(f(t+\Delta)-f(t))/\Delta$ cos\'i che $f(t+
\Delta)=f(t)+\dot{f}\Delta$. Qunidi si avr\'a:

$$ f(x,t)=\frac{x(t+\Delta)-x(t)}{\Delta}=\dot{x}+O(\Delta)$$

Quando, come in questo caso, il tempo \'e discretizzato (cio\'e
$t=n\Delta $),si ottiene un'equazione alle differenze:

$$x(t_n+1)-x(t_n)= f(x(t_n),t_n)\Delta$$

Poich\'e compare solo una differenza di primo livello (cio\'e due
livelli temporali successivi n e n+1) questa \'e un'equazione alle
differenze del primo ordine. In generale, l'ordine delle equazioni
\'e pari al numero di livelli meno uno. Ora, applicando il metodo di
Eulero all'equazione considerata, si ottiene:


\begin{itemize}
  \item Equazione differenziale:  $ x=\dot{x}=(x_{i+1}-x_i)/\Delta $
  \item Approssimazione discreta: $ x_{i+1}-x_i= x_i \Delta $
  \item Equazione alle differenze: $ x_{i+1}= (1+\Delta)x_i $
\end{itemize}


Partendo dalla condizione iniziale $x=x_0$, e iterando si ottiene:

\begin{displaymath}
\left\{
  \begin{array}{ll}
    x_1= (1+\Delta)x_0 & \hbox{} \\
    x_2= (1+\Delta)x_1=(1+\Delta)^2x_0 & \hbox{} \\
    ... & \hbox{} \\
    x_n= (1+\Delta)^n x_0 & \hbox{}
  \end{array}
\right.
\end{displaymath}


Dove $x_n=x(t_n)$. In questo modo, integrando l'equazione alle
differenze, si � ricavata la soluzione approssimata
$x_n=x_0(1+\Delta)^n $ dopo un numero n di passi (ovvero al tempo
$t_n=n\Delta$). Ricordando che lo sviluppo del fattoriale \'e
$e^\Delta = 1+\Delta+\Delta^2/2 +..+\Delta^n/n!$, la soluzione pu�
essere riscritta come:

$$x_n=(1+\Delta)^n=(1+ 1/n)^n\simeq e+ O(1/n)^2$$

dove $\Delta=1/n$.


\vskip 2.5 truecm


\textbf{Metodo centrato - schema del secondo ordine.}

Si vuole approssimare meglio la derivata per riuscire ad ottenere
una soluzione con degli errori pi\'u piccoli che nel caso
precedente. Col metodo di Eulero la derivata veniva approssimata col
rapporto incrementale, come segue:


$$\dot{x}=\frac{x(t+\Delta)-x(t)}{\Delta}+O(\Delta)=\frac{x_{i+1}-x_i}{\Delta}+O(\Delta)$$

Nel metodo centrato si approssima, invece, la derivata, nella
maniera seguente:

$$\dot{x}=\frac{x(t+\Delta)-x(t-\Delta)}{2\Delta}+O(\Delta^2)=\frac{x_{i+1}-x_{i-1}}{2\Delta}+O(\Delta^2)$$

Ovviamente con questa nuova approssimazione discreta si ottiene
un'equazione alle differenze del secondo ordine:

$$ \dot{x}={x_{i+1}-x_{i-1}\over{2\Delta}}$$

$$   x_{i+1}-x_{i-1}= 2\Delta x_i  $$

Con questo schema centrato ho ottenuto una equazione alle differenze
del secondo ordine e per poterla risolvere sono necessarie due
condizioni iniziali. Questa equazione ha due soluzioni indipendenti
di cui una approssima la nostra soluzione mentre l'altra, detta
spuria, \'e una conseguenza dell'aumento dell'ordine e non ha
controparte nel problema iniziale. Ora si vogliono calcolare queste
soluzioni.

\vskip 0.3 truecm

\textbf{Nota}: un'equazione $\ddot{x}+a\dot{x}+bx+c=0$ si dice a
coefficienti costanti se $a$ e $b$ non dipendono dal tempo. Il
termine $c$ pu\'o dipendere dal tempo. \vskip 0.3 truecm
\textbf{Teorema}: data un'equazione differenziale di ordine n a
coefficienti costanti, le sue soluzioni si possono scrivere come:

$$x(t)=\sum A_{n} e^{nt\lambda} $$
con $ \lambda_n\neq\lambda_m$. Assumendo $x_i=p^i$ si ottiene:

$$   p^2 -2\Delta p -1 =0  $$

$$ p_1 = \Delta +\sqrt{\Delta^2+1} \approx \Delta +( 1+\Delta^2/2 ) = 1+ \Delta +\Delta^2/2 $$
$$ p_2 = \Delta -\sqrt{\Delta^2+1} \approx \Delta -( 1+\Delta^2/2 ) =-1+ \Delta -\Delta^2/2 $$

La soluzione cercata sar\'a data da $x_n=A(p_1)^n+B(p_2)^n$ dove
$A(p_1)^n$ \'e un'approssimazione del secondo ordine mentre
$B(p_2)^n$ \'e errata ed occorrer\'a eliminarla annullando la
costante B. Si osserva inoltre che la soluzione spuria cambia segno
ad ogni passo, $(p_2)^n \approx (-1)^n $.

Per far partire questo schema sono necessarie le condizioni iniziali
$x_0$ e $x_1$, ed \'e la scelta di $x_1$ che determina l'ampiezza
della soluzione spuria.

$$ x_1 \simeq x_0 + \dot{x}_0 \Delta + \ddot{x}_0 \Delta^2/2 = ( 1+ \Delta + \Delta^2/2 ) x_0 $$

\begin{displaymath}
x_0=A+B\qquad  x_1=Ap_1+Bp_2
\end{displaymath}

\begin{displaymath}
 B = {x_1-p_1 x_0 \over p_2-p_1} \qquad A = x_0+{x_1-p_1 x_0 \over p_2-p_1}
\end{displaymath}
\end{document}

%\bye
